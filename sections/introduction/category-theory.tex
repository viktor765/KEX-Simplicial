\documentclass[../../main.tex]{subfiles}

\begin{document}

    Category theory is the language needed to give Simplicial/Cyclic sets a simple definition. It is also an essential part of the language of algebraic topology. It begins with the elementary definitions of Categories, functors, natural transformations and adjunctions.
        
    \begin{definition}
        A \defemph{category} $\cat{C}$ is a collection of \defemph{objects} $ob(\cat{C})$ and one of \defemph{arrows} or \defemph{morphisms} $arr(C)$ between objects. For every arrow $f$, there is an object $\domain{f}$, its \defemph{domain}, and $\codomain{f}$, its \defemph{codomain}. We say that an arrow points from $\domain{f}$ to $\codomain{f}$. For every pair of arrows $f, g$ such that $\domain{f} = \codomain{g}$ there is a composition of arrows $f \circ g$(often written omitting $\circ$) that fulfills the following axioms.
        
        \begin{enumerate}
            \item a) $\domain{fg} = \domain{g}$ \\
                b) $\codomain{fg} = \codomain{fg}$
            \item Associativity: (fg)h = f(gh) for all f,g,h where defined
            \item Identity: For every $c \in ob(C)$ there is an arrow $\idarrow[c]: c \to c$ such that $f \circ id_C = f$ and $g \circ id_c = g$ for all arrows $f, g$ where this composition is defined.
        \end{enumerate}
    \end{definition}
    
    \begin{definition}
        A (\defemph{covariant}) \defemph{functor} $\functor{F}: \cat{C} \to \cat{D}$ consists of a map of objects in $\cat{C}$ to objects in $\cat{D}$ and one map of morphisms in C to morphisms in D such that the following is fulfilled for all objects $a, b, c$ and arrows $g: a \to b, f: b \to c$.
        
        \begin{enumerate}
            \item $\domain{\functor{F}(f)} = \functor{F}(b), \codomain{\functor{F}(f)} = \functor{F}(c)$
            \item $\functor{F}(\idarrow[c]) = \idarrow[\functor{F}(c)]$
            \item $\functor{F}(fg) = \functor{F}(f) \circ \functor{F}(g)$ \label{cov}
        \end{enumerate}
        
        If \ref{cov} is reversed to $\functor{F}(fg) = \functor{F}(g) \circ \functor{F}(f)$, then $\functor{F}$ is \defemph{contravariant}.
    \end{definition}
    
    Every category $\cat{C}$ has an associated category $\cat{C^{op}}$ formed by reversing all arrows in $\cat{C}$. Every statement in $\cat{C}$ can be made in $\cat{C^{op}}$ and is then called the \defemph{dual} statement. Its truth value is preserved.
    
    Inside a given category $\cat{C}$, we often study some small, often finite, part of it. This is called a \defemph{diagram} in $\cat{C}$. Formally, an $\cat{I}$-shaped diagram $\functor{D}$ in $\cat{C}$ is a functor from an index category $\cat{I}$. If $\cat{I}$ is small, $\functor{D}$ is called a small diagram.
    
    If to a diagram $\functor{D}$ in $\cat{C}$, we add an object $e$ and arrows from $e$ to each object in $\functor{D}$ such that the resulting new diagram commutes, $e$ and the arrows form a \defemph{cone} over $\functor{D}$. A cone that is universal, in the sense that every other cone factors through it uniquely, is called a \defemph{limit} over $\functor{D}$. When a limit exist, it might not be unique. However, any other limit will be isomorphic to it. The dual notions are \defemph{cocones} and colimits. Important examples of limits are products are products, pullbacks, equalizers and terminal objects.
    
    %Insert diagrams on limits
    
    The categorical \defemph{product} can be easily defined in terms of limits. 
    
    \begin{definition}
        Given objects $\{c_i\}_{i \in I}$ in $\cat{C}$, their \defemph{product} is a limit over the diagram containing all $c_i$ and no arrows other than their identities.
    \end{definition}
    
    Examples of products is the cartesian product in $\cat{Set}$, the topological product in $\cat{Top}$ and the direct product in $\cat{Grp}$. In these categories, any set of objects has a product with these explicit constructions. The main reason of having the topological product defined the way it is(finitely many $O_i \neq X_i$) is to make it the categorical product in $\cat{Top}$. Would we not have this requirement(aka the box topology on $X$) it would merely be a cone over the discrete diagram of all $c_i$.
    
    As usual, there is a dual concept to products called a coproduct. Reverse all arrows in the definition of a product and you have a coproduct
    
    %Insert diagram on coproduct
    
    Examples of coproducts is the disjoint union in $\cat{Set}$ as well as its sibling, the disjoint union of topological spaces in $\cat{Top}$. This means that the cartesian product of topological spaces and the disjoint union of topological spaces are dual constructions. The same goes in $\cat{Set}$.
    
    One of the most important concepts in category theory is adjunctions. According to Saunders MacLane: \textit{“Adjoint functors arise everywhere.”}
    
    \begin{definition}
        Given $\functor{F}: \cat{D} \to \cat{C}$ and $\functor{G}: \cat{C} \to \cat{D}$, $\functor{F}$ is called \defemph{left adjoint} to $\functor{G}$ if for every $X \in C, Y \in D$ there is a bijection of hom-sets, $$\cat{C}(\functor{F}Y, X) \cong \cat{D}(Y, \functor{G}X)$$, which is natural in both $Y$ and $X$. $\functor{G}$ is then \defemph{right adjoint} to $\functor{F}$.
    \end{definition}
    
    An example of adjunctions are letting $\functor{G}$ be the forgetfull functor mapping $\cat{Grp} \to \cat{Set}$. Its right adjoint is then $\functor{F}: \cat{Set} \to \cat{Cat}$ which sends a set $A$ to the free group on $A$. The bijections arises from the fact that a group homomorphism is completely described by its action on the generators.
    
    Another example is the abelianazation functor $\functor{F}: \cat{Grp} \to \cat{Ab}$. It is left adjoint to the forgetfull functor $\functor{G}: \cat{Ab} \to \cat{Grp}$. This is due to the fact that the commutator subgroup $[G, G]$ of any group $G$ needs to lie in the kernel of any homomorphism into an abelian group. This means that any such homomorpishm will factor uniquely through the projection $G$ onto $G^{ab}$.
    
    For every set $X$, there are two canonical ways of defining a on it, the trivial topology and the discrete topology. These are in fact functors $\functor{F}: \cat{Set} \to \cat{Top}$. Assigning the discrete topology to $X$ by $\functor{F}$ will mean that any map \emph{from} it is continuous. Therefore we have a bijection between functions from $X$ into a given topological space $Y$ and continous maps between X and $Y$. This means that this functor is left adjoint to the forgetful functor $\functor{G}: \cat{Top} \to \cat{Set}$. If instead we let $\functor{F}$ assign the trivial topology to $X$, every function \emph{to} $X$ is continuous and by the opposite argument $\functor{F}$ is right adjoint to $\functor{G}$.
    
    The \emph{Yoneda lemma} is perhaps the most important result in elementary category theory. It tells us something essential about representable functors which will be defined. In every category $\cat{C}$ and $A \in \cat{C}$ there is a covariant functor $h^A: \cat{C} \to \cat{Set}$, often denoted $\mathrm{Hom}_\cat{C}(A, -)$. It sends objects $X \mapsto \mathrm{Hom}_\cat{C}(A, X)$ and arrows $f \mapsto f \circ -$. In fact, 
    
    \begin{equation*}
        h^-: 
        \begin{cases}
            A \mapsto h^A \\
            f \mapsto f \circ -
        \end{cases}
    \end{equation*}
    
    is a contravariant functor called the \emph{Yoneda embedding}, $h^-: \cat{C}^{op} \to \cat{Set}^\cat{C}$.
    
    \begin{definition}
        A functor $\functor{F}: \cat{C} \to \cat{Set}$ is called \defemph{representable} if there is an object $A \in \cat{C}$ such that there is a natural isomorphism $h^A \cong \functor{F}$.
    \end{definition}
    
    \begin{theorem}
        Any functor $\functor{F}: \cat{C} \to \cat{Set}$ has a bijection for all $A \in C$ 
        $$\functor{F}(A) \cong \mathrm{Nat}(h^A, \functor{F})$$ 
        natural in $A$.
    \end{theorem}
    
    \begin{proof}
        Consider the following diagram of a natural transformation $\Phi$ between $h^A$ and $\functor{F}$ applied to $A$ and an arbitrary object $X \in C$. 
        
        %insert diagram

        Let $u$ be the action of $\Phi_A$ on $\idarrow[A]$. By looking at both ways of sending $\idarrow[A]$ to $\functor{F}(X)$ it is realized that the action of $\Phi_X$ on any $f \in \mathrm{Hom}(A, X)$ is given by 

        \begin{equation}
            \Phi_X(f) = \functor{F}f(u)
        \end{equation}
        
        This means that $\Phi$ is completely determined by its action on the identity which can take on any value in $\functor{F}(A)$.
    \end{proof}

\end{document}
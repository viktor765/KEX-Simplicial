\documentclass[../../main.tex]{subfiles}

\begin{document}

    A somewhat familiar object is the topological n-simplex $\Delta_n$,
    \begin{equation}
        \Delta_n = \{(t_0, . . . , t_n)  |\sum_{i=0}^{n}t_i=1 \wedge t_i \geq 0\} \subset \mathbb{R}^{n+1}
    \end{equation}
    homeomorphic to a closed n-ball. These simplices have $n+1$ faces, e.g. a line in $\mathbb{R}^2$ with its ends as its faces. These have maps defined as follows
    
    \begin{definition}\label{simp-map}
        $d^i:\Delta_{n-1}\to \Delta_n$, $s^i:\Delta_{n+1}\to \Delta_n$
        \begin{itemize}
           \item $d^i(t_0,...,t_{n-1})=(t_0,...,t_{i-1},0,t_i,...,t_{n-1})$
           \item $s^i(t_0,...,t_{n+1})=(t_0,...,t_i+t_{i+1},...,t_{n+1})$
        \end{itemize}
    \end{definition}
    
    For a simplicial complex $X$, which is a set of simplices, the following must hold. Any face of a simplex in $X$ must be in $X$ and every intersection of any two simplices in $X$ is a face of each of them. Simplicial sets arose from the concept of simplicial complexes. Nevertheless, the two should not be confused as the first is purely algebraic and the second is both algebraic and topological.
    
    Next CW-complexes will be discussed. Simplicial complexes are similar to CW-complexes but CW-complexes are more superior for homotopy theory. A slight comparison as to how one constructs the complexes will be included at the end of the section. 
    
    A CW-complex, closure finite weakly topologized cell complex, is a somewhat complicated construction at first sight but built on simple topological concepts. First of all a cell is roughly a topological space $C$ homeomorphic to the interior n-dimensional disc, $\mathring{D}^n$. Maps can be defined so that a finite partition of the boundary is created. These images are then subcells or more commonly faces, simply cells of lower dimension. As can be understood cells are constructed inductively from points to lines to planes and so on. A cell complex $X$ is a Hausdorff space which is the disjoint union of open cells $c^n$ of which it is required that there exists a map $\varphi:\Delta\to\closure{c}^n$ such that $\varphi|(\Delta_n - \boundary{\Delta}_n )$ is a homeomorphism onto $c^n$ and $\varphi(\boundary{\Delta}_n)$ is contained in the union of cells of dimension less than n. A subcomplex $Y$ of $X$ is a union of cells such that $c^n\subset Y \Rightarrow \closure{c}^n\subset Y$. Closure finite implicates each $\closure{c}^n$ is contained in a finite subcomplex. Weakly topologized is if a subset is closed with the requirement that each intersection with each $\closure{c}^n$ is closed. That is the definition of CW-complexes according to \cite{simp-may}.
    
    The CW construction amounts to an invariant interior and boundary mapped homeomorpihcally to their respective, in a sense an abstract topological space not embedded. Examples include simplices and more generally convex polytopes.
    
    Another relevant concept connecting to CW-complexes is that of n-skeleton, $X_n$, referring to the subspace of topological space $X$ consisting of the union of cells of dimension at most n. The skeleta can be seen as a halt in the induction when creating the complex. Later it is shown that the geometric realization of simplicial sets are CW-complexes.

    The complexes and their differences can be summarized in how the simplices are glued together. From the more general CW-complex to the lesser, simplicial complexes. CW-complexes allow any glueing in as you must add the simplices in order of dimension. Simplicial complexes have further restrictions on the maps such that any face of the simplices you add must be mapped to a already existent simplex of one lesser degree. Although all of this construction occurs algebraically a so called geometric realization, a functor to topological spaces and informally the object regarded through a geometric framework, is straightforward. One simply follows the inductive process regarding each simplex as a geometric object, the use of the word glueing originates from this geometric context. This duality as both topological and algebraic notions interweave in this class of mathematics. However, for some complexes the algebraic properties are limited. A proper definition, derivation and examination of the geometric realization will follow in the next chapter.

    As is described by Maye, \cite{simp-may}, and investigated later in the report, algebraic topology could be distinguished as the study of the functors from topological spaces to that of groups invariants of homotopy type. 

    \end{document}
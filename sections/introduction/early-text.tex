\documentclass[../../main.tex]{subfiles}

\begin{document}
    The ambition of this work is a general understanding of simplicial and cyclic sets. This would eventually lead to the end goal, the $S^1$-action of cyclic sets. The mathematics are non standard on a undergraduate level and requires obtaining primary knowledge in topology, category theory and algebraic topology. Concerning the $S^1$-action, the relevant concepts are spread out through these fields. According to the structure of the report and chronologically, these concepts are introduced. 

    Topology, as is first included in the report, associates set theory to geometry. Various basic concepts have been studied, as the $S^1$ is a topological group. These include, but are not limited to, the definition of a topology, product and quotient topology as well as the definition of a Hausdorff space. More advanced is homotopy theory which is included as this is an application of simplicial and cyclic sets. For topology the main reference of the report is \cite{armstrong-basictop}.    

    Category theory emerged as an abstraction and formalization of other mathematical areas. The central concept taken from this field is functors, the $S^1$-action arises from a adjoint functor. Formalization in terms of functors produces more concise and in some sense simpler definitions, as is the case for simplicial sets. Thus a category theoretic formalization of the theory of simplicial sets has been studied motivating study of vital notions of the field of categories. Basic definitions such as diagrams, functors and of course categories but also the Yoneda lemma and comma categories are covered. The references include \cite{simp-maye} and \cite{cate-mac}.

    In the case of algebraic topology, simplices, simplicial complexes and CW-complexes will be introduced. CW-complexes are the topological analogue of simplicial sets and consist of simplices. The field was formerly known as combinatoric topology emphasizing the algorithmic and inductive construction of the complexes, which can be recognized in the section. As can be seen later in the text all algebraic topological notions are functorial, in fact the theory of categories originated from this field. For this topic \cite{simp-maye} was used as reference.

    Simplicial sets is the next stop of the report, a combinatoric model of most topological spaces through homotopy. The central definition of the section is the geometric realization of a simplicial set, a functor to topological spaces. Theorems studied and included are the geometric realization of a simplicial set is Hausdorff and also a CW-complex and mentioned is that the realization commutes with the product of simplicial sets. \cite{luk-simp} was instructive and used as a reference for simplicial sets. Cyclic sets, permitting the morphism of cyclic permutation, are a generalization of simplicial sets further simplifying some topological objects. Here the geometric realization is once again of main importance culminating in the $S^1$-action on a cyclic set. \cite{loday-cyclic} was used as reference for cyclic sets.

    Lastly we would to personally thank our advisor Tilman Bauer for always being available and Lars Svensson for introducing these fields and inspiring to dive into the theory of categories and beyond.
    
\end{document}
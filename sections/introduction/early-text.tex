\documentclass[../../main.tex]{subfiles}

\begin{document}

    Simplicial sets arose from the concept of simplicial complexes, many of the results are similar in proof and statement. Nevertheless, the two should not be confused as the first is purely algebraic with a geometric realization as a CW-complex, a more general topological complex than the simplicial. 
    
    Topology, as is first introduced, relates set theory and geometry. It is the study of continuous maps and the properties for which topological spaces are invariant and in particular invertible continuous maps, such that certain properties of the spaces are invariant. For topology the main reference is \cite{armstrong-basictop}.
    
    In the case of algebraic topology a few constructions of complexes will be introduced. The field used to be called combinatoric topology emphasizing the algorithmic approach of construction which can be recognized in the section. However, the correspondence to more manageable branches of mathematics, especially group theory, led to the name change. As can be seen later in the text all algebraic topological notions are functorial, in fact the theory of categories originated from this field. 
    
    Category theory emerged as an abstraction and formalization of other mathematical areas. This led to the fundamental conception of functors associating various branches of mathematics and their structures with one another. 
    
    
    
    Definitions as most commonly stated in their corresponding area of mathematics follow below. 
    
\end{document}
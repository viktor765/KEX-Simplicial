\documentclass[../../main.tex]{subfiles}

\begin{document}

    Simplicial sets arose from the concept of simplicial complexes, many of the results are similar in proof and statement. Nevertheless, the two should not be confused as the first is purely algebraic and the second is both algebraic and topological. Simplicial sets are constructed through ordered sets and order preserving maps, simple but abstract as we shall see is the case for many complex topological structures. Cyclic sets, permitting the morphism of cyclic permutation, are a generalization of simplicial sets further simplifying some topological objects.
    
    Topology, as is first introduced, relates set theory and geometry. It is the study of continuous maps and the properties for which topological spaces are invariant and in particular invertible continuous maps, such that certain properties of the spaces are invariant. For topology the main reference is \cite{armstrong-basictop}. Relevant concepts will be introduced and in some cases a category theoretic formalization will be included. 
    
    In the case of algebraic topology a few constructions of complexes will be introduced. The field used to be called combinatoric topology emphasizing the algorithmic approach of construction which can be recognized in the section. However, the correspondence to more manageable branches of mathematics, especially group theory, led to the name change. As can be seen later in the text all algebraic topological notions are functorial, in fact the theory of categories originated from this field. 
    
    Category theory emerged as an abstraction and formalization of other mathematical areas. This led to the fundamental conception of functors which associate various branches of mathematics and their structures with one another. As was mentioned this is a concept derived from algebraic topology. Functorial formalization is a major advantage when using the theory as it makes definitions, as is the case for simplicial sets, and results in general more concise and in some sense simpler. 
    
    Definitions as most commonly stated in their corresponding area of mathematics follow below. 
    
\end{document}
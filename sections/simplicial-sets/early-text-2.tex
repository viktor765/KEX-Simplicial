\documentclass[../../main.tex]{subfiles}
\begin{document}
    In this chapter a composition of the aforementioned theory will lead to defining simplicial sets and cyclic sets. In combination with the ubiquitous category formalization a more historic and hopefully more intuitive approach will be included of the first definition. Following that a treatise of the geometric realization of the sets, a functor to topological spaces, will integrate and motivate the theory of the prior chapter. For cyclic sets, of which the theory is based heavily on the simplicial, an analogues treatise will take place, highlighting the key differences. Additionally a few examples will be included, demonstrating the strength and versatility of the theory of simplicial and cyclic sets.
\end{document}
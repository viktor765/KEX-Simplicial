\documentclass[../../main.tex]{subfiles}

\begin{document}

    A cyclic set is a presheaf on the cyclic category in parallel with  simplicial for $\Delta$. Another definition is as a functor $X:\Delta \cat{C^{op}} \to \cat{sSet}$. These multiple definitions are recognized and similar to the ones of simplicial sets. As defined by \cite{loday-cyclic} see below.
    
    \begin{definition}
        The cyclic category $\Delta \cat{C}$ has objects $[n], \;n\in N$ and morphisms generated by face $\delta$ and degeneracy $\sigma$ maps as the simplicial. In addition a cyclic automorphism $\tau_n:[n]\to [n]$ is included necessitating the following relations.
        \begin{enumerate}
            \item $\tau _n\delta_i=\delta_{i-1}\tau_{n-1},\; 1 \leq i\leq n$
            \item $\tau _n \sigma_i=\sigma_{i-1}\tau_{n+1},\; 1 \leq i\leq n$
            \item $\tau_{n}^{n+1}=id$
        \end{enumerate}
    \end{definition}

    For the opposite category $\Delta \cat{C^{op}}$ the maps $\delta_i^*, \; \sigma_i^*,\;\tau_n$ are denoted $d_i,\; s_i, \; t_n$ respectively. Similarly any morphism can be expressed uniquely as a morphism in $\Delta$ and an element in a cyclic group $C_n$ of order $n+1$ motivating the notation $\Delta C$. Furthermore as is shown in Loday, the category $\Delta \cat{C}$ contains $\Delta$ as a subcategory and utilizing a forgetful functor gives the underlying simplicial set or object. 

    The circle is an obvious choice of study when introducing geometric realization in general. Several models exist and it is straightforward for simplicial sets, however, cyclic sets excel. The action of the corresponding cyclic group is somewhat compatible with simplicial structure and induces an action of the topological group $S^1$. This is summarized in the following theorem from \cite{loday-cyclic}.

    \begin{theorem}\label{hardstuff}
        The geometric realization, of a cyclic set $X$, denoted $\geom[X]$ of its underlying simplicial space
        \begin{itemize}
            \item $X$ is endowed with a canonical action of the circle $S^1$
            \item $X \to \geom[X]$ is a functor from cyclic spaces to $\cat{S^1-spaces}$
        \end{itemize}
    \end{theorem} %kolla extensiva utdraget

    A proof of this will be included later. In the case of cyclic sets the geometric realization is based on the underlying simplicial set, however, this requires some additional groundwork for all natural and intuitive notions to be maintained. In terms of category theory, the forgetful to the underlying simplicial set permits a left adjoint functor described as follows according to \cite{loday-cyclic}. 

    \begin{definition}
        Let X be a simplicial space and let F(Y) be defined by:
        $F(Y)_n=C_n\times Y_n, \qquad f_*(g,y)=(f_*(g), (g^*(f)_*(y)))$
        $h^*(g,y) = (hg,y) $ for $f$ in $\Delta^{op}$, g and h $\in C_n$, $y\in Y_n$
    \end{definition}

    In conjunction with the following lemmas from \cite{loday-cyclic} a proof for theorem \ref{hardstuff} 

    \begin{lemma}
        \begin{itemize}
            \item F(Y) is a cyclic space
            \item if X is a cyclic space then the evaluation map ev: F(X) $\to$ X, given by (g,x) $\mapsto$ $g_*(x)$ is a morphism of cyclic spaces
            \item the functor F : is left adjoint to the forgetful functor.
        \end{itemize}
    \end{lemma}

    \begin{lemma}
        For any simplicial space X the map $(p_1,p_2)\; : \;\geom[F(X)]\to \geom[C]\times \geom[X]= S^1\times \geom[X]$ is a homeomorphism. 
    \end{lemma}

    \begin{lemma}
        $\geom[\cat{C}]= S^1$ and $\zeta : \geom[\cat{C}]\times \geom[\cat{C}]\to \geom[\cat{C}]$
    \end{lemma}

\end{document}
\documentclass[../../main.tex]{subfiles}

\begin{document}

    A cyclic set is a presheaf on the cyclic category in parallel with  simplicial for $\Delta$. Another definition is as a functor $X:\Delta C^{op}\to sSet$. These multiple definitions are recognized and similar to the ones of simplicial sets. As defined by Loday see below.
    
    \begin{definition}
        The cyclic category $\Delta C$ has objects $[n], \;n\in N$ and morphisms generated by face $\delta$ and degeneracy$s$ maps as the simplicial. However, a cyclic automorphism $\tau_n:[n]\to [n]$ is included neccesiting the following relations.
        \begin{enumerate}
            \item $\tau _n\delta_i=\delta_{i-1}\tau_{n-1},\; 1 \leq i\leq n$
            \item $\tau _n \sigma_i=\sigma_{i-1}\tau_{n+1},\; 1 \leq i\leq n$
            \item $\tau_{n}^{n+1}=id$
        \end{enumerate}
    \end{definition}

    For the opposite category $\Delta C^{op}$ the maps $\delta_i^*, \; \sigma_i^*,\;\tau_n$ are denoted $d_i,\; s_i, \; t_n$ respectively. Similarly any morphism can be expressed uniquely as a morphism in $\Delta$ and an element in a cyclic group $C_n$ of order $n+1$ motivating the notation $\Delta C$. Furthermore as is shown in Loday, the category $\Delta C$ contains $\Delta$ as a subcategory and utilizing a forgetful functor gives the underlying simplicial set or object. 

    The circle is an obvious choice of study when introducing geometric realization in general. Several models exist and it is straightforward for simplicial sets, however, cyclic sets excel. The action of the corresponding cyclic group is somewhat compatible with simplicial structure and induces an action of the topological group $S^1$. This is summarized in the following theorem from Loday.

    \begin{theorem}
        The geometric realization, of a cyclic set $X$, denoted $\geom[X]$ of its underlying simplicial space
        \begin{itemize}
            \item $X$ is endowed with a canonical action of the circle $S^1$
            \item $X \to \geom[X]$ is a functor from cyclic spaces to $S^1-spaces$
        \end{itemize}
    \end{theorem}

    

\end{document}
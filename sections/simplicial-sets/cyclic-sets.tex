\documentclass[../../main.tex]{subfiles}

\begin{document}

    The definition of a cyclic set is as a functor $X:\Delta \cat{C^{op}} \to \cat{Set}$. This is recognized and similar to the one of simplicial sets. The cyclic category as defined by \cite{loday-cyclic} see below.
    
    \begin{definition}
        The cyclic category $\Delta \cat{C}$ has objects $[n], \;n\in N$, like $\Delta$, and morphisms generated by coface and codegeneracy maps $\delta_i$ and $\sigma_i$ maps as in $\Delta$ but also maps of an additional type. These are the cyclic automorphisms $\tau_n:[n]\to [n]$ described by the following figure.
    \end{definition}

    \begin{figure}[H]
        \[
            \begin{tikzcd}[row sep = 1ex, column sep = 10em]
                0 \arrow[dr] & 0 \\
                1 \arrow[dr] & 1 \\
                :  & 2 \\
                n-1 \arrow[dr] & :\\
                n \arrow[uuuur] & n            
            \end{tikzcd}
        \]
        \caption{Illustration of the cyclic morphism $\tau$. As can be seen this is not a simplicial map.}
    \end{figure}

    Similarly as $\Delta$ any morphism in $\Delta C$ can be expressed uniquely as a composite of a morphism in $\Delta$ and an element in a cyclic group $C_n$ of order $n+1$ motivating the notation $\Delta C$. For the opposite category $\Delta \cat{C^{op}}$ the maps $\delta_i^*, \; \sigma_i^*,\;\tau_n$ are denoted $d_i,\; s_i, \; t_n$ respectively. And the inclusion of $t$ necessitates the following additional relations for a cyclic set.
    
    \begin{enumerate}
        \item $d_it _n = t_{n-1}d_{i-1},\; 1 \leq i\leq n, \; d_0t_n=d_n$
        \item $s_i t _n = t_{n+1}s_{i-1},\; 1 \leq i\leq n, \; s_0t_n=t_{n+1}^2s_n$
        \item $(t_{n})^{n+1}=id$
    
    \end{enumerate}
    
    Note that $d_0t_n=d_n$ and $s_0t_n=t_{n+1}^2s_n$ are a consequence of the other relations. 

    As is shown in \cite{loday-cyclic}, the category $\Delta \cat{C}$ contains $\Delta$ as a subcategory. This means there is an embedding functor $\functor{E}: \Delta \to \Delta C$. Composing a cyclic set $X: \Delta C^{op} \to Set$ with this embedding will produce the \emph{underlying} simplicial set $X: \Delta^{op} \to Set$. The composition with $\functor{E}$ is the forgetful functor $\functor{G}: \cat{cSet} \to \cat{sSet}$. Now the geometric realization $\geom[X]$ of a cyclic set $X$ can be defined to be the realization of the underlying simplicial set.

    \cite{loday-cyclic} also shows that $\functor{G}$ has a left-adjoint functor $\functor{F}: \cat{sSet} \to \cat{cSet}$ which has the following property in conjunction with the geometric realization.

    \begin{equation*}
        \geom[\functor{F}(X)] \cong S^1 \times \geom[X]
    \end{equation*}

    Especially this means that 
    
    \begin{equation}\label{loop-spaces}
        \geom[\functor{F}(\Delta^n)] \cong S^1 \times \geom[\Delta^n].
    \end{equation}

    The circle is an obvious choice of study when introducing geometric realization in general. Several models exist and they are straightforward to find with simplicial sets. One example is letting $n = 0$ in \eqref{loop-spaces} which is a cyclic set(and then also a simplicial). However, cyclic sets excel. The action of the corresponding cyclic group is somewhat compatible with simplicial structure and induces an action of the topological group $S^1$. This is summarized in the following theorem from \cite{loday-cyclic}.

    \begin{theorem}\label{main-result}
        The following is true about the geometric realization of a cyclic set $X$.
        \begin{itemize}
            \item $\geom[X]$ is endowed with a canonical action of the circle $S^1$
            \item $X \to \geom[X]$ is a functor from cyclic sets to $S^1$-spaces
        \end{itemize}
    \end{theorem}

    A proof of how the cyclic action arises from the adjoint functor will be included, however, a full proof of theorem \ref{main-result} can be found in \cite{loday-cyclic}.

    Before proving \ref{main-result}, a quick reconsideration of adjoint functors is needed. 
    Of adjointness, an equivalent definition to \ref{def-adjoint} is by so called \defemph{unit-counit adjunction}. The proof of this equivalence is given in \cite{cate-mac}.

    \begin{definition}
        Two functors $\functor{F}: \cat{C} \to \cat{D},\, \functor{G}: \cat{D} \to \cat{C}$ are \defemph{adjoint} if there exists natural transformations $\varepsilon: \functor{FG} \to \mathrm{id}_\cat{D}$ and $\eta: \mathrm{id}_\cat{C} \to \functor{GF}$ such that $\varepsilon \functor{F} \circ \functor{F}\eta = \idarrow[\functor{F}]$ and $\functor{G}\varepsilon \circ \eta \functor{G} = \idarrow[F]$. In other words the following diagrams commute for every $X$.
    \end{definition}

    \begin{figure}[H]
        \begin{subfigure}[b]{0.5\textwidth}
            \[
                \begin{tikzcd}[row sep = large]
                    \functor{FGF}X \arrow[d,"\mathlarger{\eta}_{\functor{F}X}", swap] & \functor{F}X \arrow[l,"\functor{F}\mathlarger{\varepsilon}_X", swap]\\
                    \functor{F}X \arrow[ur, equal]
                \end{tikzcd}
            \]
        \end{subfigure}%
        ~
        \begin{subfigure}[b]{0.5\textwidth}
            \[
                \begin{tikzcd}[row sep = large]
                    \functor{GFG}X \arrow[r,"\functor{G} \mathlarger{\varepsilon}_X"] & \functor{G}X \\
                    \functor{G}X \arrow[u,"\mathlarger{\eta}_{\functor{G}X}"] \arrow[ur, equal]
                \end{tikzcd}
            \]
        \end{subfigure}
    \end{figure}

    \begin{definition}
        Let X be a simplicial space and let F(Y) be defined by:
        $F(Y)_n=C_n\times Y_n, \qquad f_*(g,y)=(f_*(g), (g^*(f)_*(y)))$
        $h^*(g,y) = (hg,y) $ for $f$ in $\Delta^{op}$, g and h $\in C_n$, $y\in Y_n$
    \end{definition}

    Interpret $\functor{G}$ as the forgetful functor from $\cat{cSet}$ to $\cat{sSet}$ and $\functor{F}$ its left adjoint. Applying the geometric realization functor on the left diagram the following is obtained, with the action on $x \in \geom[X]$ drawn in.

    \begin{figure}[H]
        \[ 
            \stackinset{l}{10ex}{b}{13ex}{%
                \scalebox{.8}
                {% 
                    \begin{tikzcd}[row sep=11ex, column sep = 11ex, ampersand replacement=\&]
                        (1,x)\arrow[mapsto]{r} \& x \arrow[dl,equal] \\
                        x \arrow[u,mapsto]
                    \end{tikzcd}
                }
            }{% 
                \begin{tikzcd}[row sep = 25ex, column sep = 25ex, ampersand replacement=\&]
                    S^1 \times \geom[X] \arrow[r,"|\functor{G} \mathlarger{\varepsilon}_X|"] \& \geom[X] \\
                    \geom[X] \arrow[u,"|\mathlarger{\eta}_{\functor{G}X}|"] \arrow[ur, equal]
                \end{tikzcd}
            }
        \]
    \end{figure}

\end{document}
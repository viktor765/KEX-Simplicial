\documentclass[../../main.tex]{subfiles}

\begin{document}

    The definition of a cyclic set is as a functor $X:\Delta \cat{C^{op}} \to \cat{Set}$. This is recognized and similar to the one of simplicial sets. The cyclic category as defined by \cite{loday-cyclic} see below.
    
    \begin{definition}
        The cyclic category $\Delta \cat{C}$ has objects $[n], \;n\in N$, like $\Delta$, and morphisms generated by face $\delta$ and degeneracy $\sigma$ maps as the simplicial. In addition a cyclic automorphism $\tau_n:[n]\to [n]$ is included.
    \end{definition}

    Similarly as $\Delta$ any morphism in $\Delta C$ can be expressed uniquely as a composite of a morphism in $\Delta$ and an element in a cyclic group $C_n$ of order $n+1$ motivating the notation $\Delta C$. For the opposite category $\Delta \cat{C^{op}}$ the maps $\delta_i^*, \; \sigma_i^*,\;\tau_n$ are denoted $d_i,\; s_i, \; t_n$ respectively. And the inclusion of $t$ necessitates the following additional relations for a cyclic set.
    \begin{enumerate}
        \item $d_it _n = t_{n-1}d_{i-1},\; 1 \leq i\leq n, \; d_0t_n=d_n$
        \item $s_i t _n = t_{n+1}s_{i-1},\; 1 \leq i\leq n, \; s_0t_n=t_{n+1}^2s_n$
        \item $(t_{n})^{n+1}=id$
    \end{enumerate}
    Note that $d_0t_n=d_n$ and $s_0t_n=t_{n+1}^2s_n$ are a consequence of the other relations. 
    \begin{figure}[H]
        \[
        \begin{tikzcd}[row sep = 1ex, column sep = 10em]
            0 \arrow[dr] & 0 \\
            1 \arrow[dr] & 1 \\
            :  & 2 \\
            n-1 \arrow[dr] & :\\
            n \arrow[uuuur] & n            
        \end{tikzcd}   
        \]
        \caption{Illustration of the cyclic morphism $\tau$, as can be seen this is not a simplicial map}
    \end{figure}


    Furthermore as is shown in \cite{loday-cyclic}, the category $\Delta \cat{C}$ contains $\Delta$ as a subcategory and utilizing a forgetful functor gives the underlying simplicial set or object. Another property of the cyclic category is that it is isomorphic to its opposite category \cite{loday-cyclic}.

    The circle is an obvious choice of study when introducing geometric realization in general. Several models exist and it is straightforward for simplicial sets, however, cyclic sets excel. The action of the corresponding cyclic group is somewhat compatible with simplicial structure and induces an action of the topological group $S^1$. This is summarized in the following theorem from \cite{loday-cyclic}.

    \begin{theorem}\label{hardstuff}
        The geometric realization, of a cyclic set $X$, denoted $\geom[X]$ of its underlying simplicial space
        \begin{itemize}
            \item $X$ is endowed with a canonical action of the circle $S^1$
            \item $X \to \geom[X]$ is a functor from cyclic spaces to $\cat{S^1-spaces}$
        \end{itemize}
    \end{theorem} %check book

    A proof of how the cyclic action arises from the adjoint functor will be included, however, a full proof of theorem \ref{hardstuff} can be found in \cite{loday-cyclic}. In the case of cyclic sets the geometric realization is based on the underlying simplicial set, this does however require some additional groundwork for all natural and intuitive notions to be maintained. In terms of category theory, the forgetful functor to the underlying simplicial set permits a left adjoint functor described as follows according to \cite{loday-cyclic}. 

    \begin{definition}
        Let X be a simplicial space and let F(Y) be defined by:
        $F(Y)_n=C_n\times Y_n, \qquad f_*(g,y)=(f_*(g), (g^*(f)_*(y)))$
        $h^*(g,y) = (hg,y) $ for $f$ in $\Delta^{op}$, g and h $\in C_n$, $y\in Y_n$
    \end{definition} 
    
    \begin{lemma}
        \begin{itemize}
            \item F(Y) is a cyclic space
            \item if X is a cyclic space then the evaluation map ev: F(X) $\to$ X, given by (g,x) $\mapsto$ $g_*(x)$ is a morphism of cyclic spaces
            \item the functor F : is left adjoint to the forgetful functor.
        \end{itemize}
    \end{lemma}

    \begin{lemma}
        For any simplicial space X the map $(p_1,p_2)\; : \;\geom[F(X)]\to \geom[C]\times \geom[X]= S^1\times \geom[X]$ is a homeomorphism. 
    \end{lemma}

    \begin{lemma}
        $\geom[\cat{C}]= S^1$ and $\zeta : \geom[\cat{C}]\times \geom[\cat{C}]\to \geom[\cat{C}]$
    \end{lemma}

\end{document}
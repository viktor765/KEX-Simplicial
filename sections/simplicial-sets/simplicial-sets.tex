\documentclass[../../main.tex]{subfiles}

\begin{document}
          
    Simplicial sets can be seen as a generalization of directed multigraphs. Calling the vertices in a multigraph 0-simplices and the edges 1-simplices, each 1-simplex $k_1$ is assigned two 0-simplices $d_0k_1$ and $d_1k_1$, the head and tail vertex of $k_1$ respectively. The maps $d_i$ are called the \emph{face maps} for 1 simplices. This raises the question what exactly an $n$-simplex should be. An $n$-simplex from geometry has $n+1$ vertices and faces, which can all be considered simplices too.

    \begin{example}[2-simplex]
        A $2$-simplex is a triangle, which has three faces(sides) and three vertices, that we will assume are ordered. The order is arbitrary. Each face can be uniquely identified with the vertex opposite of it, which induces an order on the faces too. Having a set of triangles with vertices denoted by $\{0..2\}$ we therefore have a face map $d_i$ sending every 2-simplex to its $i$'th edge which is a 1-simplex. This can in turn be sent to its $2$ faces which are vertices of a triangle. 
        
        Therefore the situation could modeled as follows. There are three sets $X_2$, $X_1$ and $X_0$ of $2$-, $1$- and $0$-simplices with maps $d_i: X_n \to X_{n-1}, i \in \{0,.., n\}$ between them. The structure is realized by observing that $d_0d_1 = d_0d_0$, $d_1d_2 = d_1d_1$ and $d_0d_2 = d_1d_0$ must hold as the vertex at the head of an edge must be at the tail of the following. In other words, these relations arise from the fact that any vertex can be reached in two ways because it is part of two sides.
    \end{example}

    In a \emph{simplicial set} $X$(to be defined) there is an $X_n$ for all $n \in \mathbb{N}$, even when we are not modeling any simplices of dimension higher than $n$ for some $n$. As long as there is at least one simplex of lower dimension these will not be empty due to the \emph{degeneracy maps} $s_i$. The following definition will make clear what they are and also describe the structure of a simplex in a simplicial set.

    \begin{definition}\label{simplicial-set}
        A \defemph{simplicial set} $X$ is a collection of sets $\{X_n\}$ indexed over $\mathbb{N}$ together with maps $d_i: X_n \to X_{n-1}$ and $s_i: X_in\to X_{n+1}$ where $0 \leq i \leq n$, subject to the following axioms called the \emph{simplicial identities}.
        
        \begin{enumerate}
            \item $d_id_j = d_{j-1}d_i$ if $i < j$
            \item $s_is_j = s_{j+1}s_i$ if $i \leq j$
            \item $d_is_j = s_{j-1}d_i$ if $i < j$
            \item $d_js_j = d_{j+1}s_j = id$
            \item $d_is_j = s_jd_{i-1}$ if $i > j+1$
        \end{enumerate}
    \end{definition}

    The elements of $X_n$ are called $n$-simplices. The $d_i$ and $s_i$ are called \defemph{face maps} and \defemph{degeneracy maps}. A simplex that can be written $s_ix_n$ for some $x_n \in X_n$ is called a \defemph{degenerate simplex}.
    
    This definition of a simplicial set is quite cumbersome. Luckily, there is a much more elegant definition in the language of category theory that also reveals a lot of the structure of simplicial sets. We will first have to introduce the category $\Delta$.

    \begin{definition}[The simplex category $\Delta$]
        The category $\Delta$ consists of all finite ordinals $[n] = \{0, .., n\}$ with arrows being all monotone(order preserving) maps.
    \end{definition}

    For $[n]$ and $[n-1]$ in $\Delta$ there are monotone maps $d^i: [n-1] \to [n]$, the monotone injection that misses $i \leq n$, and $s^i: [n] \to [n-1]$, the monotone surjection sending $i$ and $i+1$ to $i \leq n-1$. 

    %insert figure of d^i and s^i

    An important property of $\Delta$ is the following property of these.

    \begin{proposition}
        Any morphism $f: [m] \to [n]$ in $\Delta$ can be written uniquely on the form

        \begin{equation*}
            f = d^{i_1}...d^{i_k}s^{j_1}...s^{j_l}
        \end{equation*},

        where $i_k < ... < i_1$ and $j_1 < ... < j_l$ and $n - m = l - k$
    \end{proposition}

    \begin{proof}
        Let $i_1, ..., i_k$ be the elements of $[n] \setminus \Ima f$ in reverse order and $j_1, ..., j_l$ the elements of $\{j \in [m]: f(j) = f(j+1)\}$ in order. It is realized that this construction has the property $f = d^{i_1}...d^{i_k}s^{j_1}...s^{j_l}$ and is also unique.
    \end{proof}

    The following definition of a simplicial set is equivalent to $\ref{simplicial-set}$.

    \begin{definition}
        A \defemph{simplicial set} is a contravariant functor from $\Delta$ to $\cat{Set}$.
    \end{definition}

    In even shorter terms: a simplicial set $X$ is a presheaf on $\Delta$. It can also be said that it is a covariant functor from $\Delta^{op}$ to $\cat{Set}$. More generally replacing $\cat{Set}$ with an arbitrary category of "objects", $X$ is a \defemph{simplicial object}.
    
    Geometric realization, as was mentioned in the example \ref{functor_exmp}, is a functor from $\mathbf{sSet}\to \mathbf{CGHaus}$. This entails that a simplicial set $X$ can be used to build a topological space $\geom[X]$ simply by interpreting each n-simplex in $X$ for all n as a corresponding copy of the standard topological n-simplex $\geom[\Delta^n]$. The topological parts are then glued together specified by face and degeneracy maps, forming the whole space. The aforementioned can be formalized as a colimit taken over n-simplex category of $X$, $\geom[X]= \lim_{\Delta^n\to X}\geom[\Delta^n]$. An algebraic topological definition follows.

    \begin{definition}
        For a simplicial set $X$, given the discrete topology to each $X_n$ and with the standard n-dimensional simplex in $\mathbb{R}^{n+1}$, the \defemph{geometric realization} $\geom[X]$ of $X$ is defined as
        \[\geom[X]=\coprod_{n=0}^{\infty} X_n \times \geom[\Delta^n]/\sim.\]
        $\geom[X]$ is equipped with the quotient topology and the equivalence relation $(\sim)$ defined as 
        \begin{itemize}
            \item $(d_ik_{n},u_{n-1})\sim(k_{n}, d^iu_{n-1})$
            \item $(s_ik_n, u_{n+1})\sim(k_n, s^iu_{n+1})$
        \end{itemize}
    \end{definition}
    
    \begin{proposition}
        The geometric realization of a simplicial set $X$ is Hausdorff.
    \end{proposition}

    \begin{proof}
        Any given point $x \in \geom[X]$ has a unique non-degenerate representative, in some simplex $k_n$. Make a small open neighborhood of $x$ in the interior of $k_n$ and call it $N_k(x)$. Assuming $x \neq y$ we can also create a disjoint such $N_l(y)$.
        
        We will create open neighborhoods of $x$, $N_n(x)$ for every dimension that are closed under the equivalence relation when restricting ourselves to simplices of dimension $\leq n$. This is done by including all such $N_{n'}(x)$ of $n' < n$ and also all equivalent degenerate points in $N_n(x)$.
        
        Assume the construction is made up to dimension $n$. Via $\sim$ there are subsets of non-degenerate $(n+1)$-simplices $k_{n+1}$ identified with $N_n$ in $\geom[X]$. Now for any such $k_{n+1}$, we can extend $N_n(x)$ to an open subset of $k_{n+1}$ by stretching it slightly towards the barycenter. If $y \in \interior{k_{n+1}}$ then this neighborhood and $N_{n+1}(y)$ can be made small enough not to intersect each other. Also, if $N_{n}(y)$ is identified with another face of $k_{n+1}$ they will not intersect.

        If $k_{n+1}$ is a degenerate simplex we will instead let the neighborhood of $x$ in it to be the preimage of the projection onto $k_n$. This is important as all these points are in the identification of $N_n(x)$.

        Now take $N_{n+1}(x)$ to be the union of all the neighborhoods of $N_n(x)$ created in this fashion for every $k_{n+1}$ considered. It is disjoint from $N_{n+1}(y)$. Let

        \begin{equation*}
            N(x) = \Pi_{n \geq 0} N_n(x)
        \end{equation*}

        and likewise create $N(y)$. $N(x)$ and $N(y)$ are saturated open sets that are disjoint and so their projections on $\geom[X]$ will be open and disjoint.
    \end{proof}
    
    \tikzset{->-/.style={decoration={
    markings,
    mark=at position .5 with {\arrow{>}}},postaction={decorate}}}

    \begin{example}
        A few examples of how identifying sides and points in a square can result in complex topological structures. In the squares below, figure \ref{fig:squares}, the sides, faces, have been identified, arrowheads pointing in the same direction indicate the sides being "glued" or identical. Below is a diagram realizing as a torus, one arrow, identification, results in a cylinder and the other of the glueing of the top and bottom. If one arrow were to be turned around it would realize as a Klein bottle, see below.
        \begin{figure}[H]
            \begin{subfigure}[b]{0.3\textwidth}
                \begin{tikzpicture}[thick, scale=2]
                \node[circle, draw=black, fill=black, inner sep=0pt, minimum size=1pt,name=0] at (0,0) {};
                \node[xshift=-2mm] at (0.west) {d};
                
                \node[circle, draw=black, fill=black,inner sep=0pt, minimum size=1pt, name=1] at (0,1) {};
                \node[xshift=-2mm] at (1.west) {a};
                
                \node[circle, draw=black, fill=black,inner sep=0pt, minimum size=1pt, name=3] at (1.5,0) {};
                \node[xshift=2mm] at (3.east) {b};
                
                \node[circle, minimum width=1pt, draw=black, fill=black,inner sep=0pt, minimum size=1pt, name=2] at (1.5,1) {};
                \node[xshift=2mm] at (2.east) {c};
                
                
                \draw[dashed] (0) -- (2);
                \draw[->-] (3) -- (0);
                \draw[->-] (0) -- (1);
                \draw[->-] (2) -- (1);
                \draw[->-] (3) -- (2);
            \end{tikzpicture}
        \end{subfigure}
        \qquad
        \begin{subfigure}[b]{0.3\textwidth}
            \begin{tikzpicture}[thick, scale=2]
                \node[circle, draw=black, fill=black, inner sep=0pt, minimum size=1pt,name=0] at (0,0) {};
                \node[xshift=-2mm] at (0.west) {d};
                
                \node[circle, draw=black, fill=black,inner sep=0pt, minimum size=1pt, name=1] at (0,1) {};
                \node[xshift=-2mm] at (1.west) {a};
                
                \node[circle, draw=black, fill=black,inner sep=0pt, minimum size=1pt, name=3] at (1.5,0) {};
                \node[xshift=2mm] at (3.east) {b};
                
                \node[circle, minimum width=1pt, draw=black, fill=black,inner sep=0pt, minimum size=1pt, name=2] at (1.5,1) {};
                \node[xshift=2mm] at (2.east) {c};
                
                
                \draw[dashed] (0) -- (2);
                \draw[->-] (3) -- (0);
                \draw[->-] (1) -- (0);
                \draw[->-] (2) -- (1);
                \draw[->-] (3) -- (2);
            \end{tikzpicture}
        \end{subfigure}
        \caption{A torus(left) and a Klein bottle(right)}
        \label{fig:squares}
        \end{figure}
        Should all the vertices, endpoints, be identified the diagram realizes as a sphere. In figure \ref{fig:squares} the degenerate simplices are omitted as is often the case. On a side note the simplicial complex would have needed additional 2-simplices as they would have need to remain being triangles after the realization as a torus.
    \end{example}
    Product of simplicial sets
    Geometric realization is CW-complex
    Product of geometric realization

\end{document}